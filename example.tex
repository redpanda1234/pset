\documentclass{pset}

% ================================================================== %
%                                                                    %
%                              Document                              %
%                                                                    %
% ================================================================== %

% ----------------------- Header formatting ------------------------ %

\name{}
\class{Math 80}
\prof{Pippenger}
\assignment{1}
\duedate{3/27/2018}
\dueday{Tuesday}
\problems{1, 2, 3, 4}
\acknowledgements{{}, {}, {}, {}}
\onTime{0}
\season{Spring}

\lfoot{Due Tuesday, March 27th 2018}

\comments{\indent For Homework assignments, you may work with others,
  but you must write up the solution you submit yourself, and it must
  reflect your own understanding of the solution. You may use any
  resources (books, papers, the Internet) in working on homework
  assignments, but in accordance with scholarly conventions, you
  should acknowledge any sources you benefit from.

  \indent For homework assignments, in solving a particular problem or
  part of a problem, you may use results stated in any previous
  problem or parts of problems (in the same or earlier assignments),
  even if you have not solved those previous problems. }

\begin{document}

% --------------------------- Problem 1 ---------------------------- %

  \section*{Problem 1.}
    Suppose $y_1(x)$ and $y_2(x)$ are two solutions of
    \[
      y''(x) + P(x)y'(x) + Q(x)y(x) = 0.
    \]
    Show that if the Wronskian
    \[
      W(x) = \det
      \begin{pmatrix}
        y_1(x) & y_2(x) \\
        y'_1(x) & y'_2(x)
      \end{pmatrix}
    \]
    does not vanish for $x = 0$, then it does not vanish for any value
    of $x$. (Hint: Show that $W'(x) + P(x)W(x) = 0$).

  \hrulefill

  \section*{Solution:}

  \clearpage

% --------------------------- Problem 2 ---------------------------- %

  \section*{Problem 2.}
    \newcommand{\GG}{G(r, \theta, t)}
    For the wave equation
    \[
      \pd[2]{}{r} \GG + \frac 1 r \pd{}{r} \GG + \frac 1 {r^2}
      \pd[2]{}{\theta} \GG - \frac 1 {v^2} \pd[2]{}{t} \GG = 0
    \]
    in polar coordinates, look for a solution $\GG = R(r) \Theta
    (\theta) T(t)$ that is not necessarily radially symmetric. Find
    the solutions for $T(t)$ and $\Theta(\theta)$, and express $R(r)$
    in terms of the solutions of the equation
    \[
      y''(s) + \frac 1 s y'(s) + \pn{1 - \frac{n^2}{s^2}} y(x) = 0
    \]
    which is known as ``Bessel's equation of order $n$.'' Explain why
    $n$ must be an integer.

  \hrulefill

  \section*{Solution:}

  \clearpage

% --------------------------- Problem 3 ---------------------------- %

  \section*{Problem 3.}
    Find two linearly independent power-series solutions of
    \[
      y''(x) + xy'(x) + y(x) = 0
    \]

  \hrulefill

  \section*{Solution:}

  \clearpage

% --------------------------- Problem 4 ---------------------------- %

  \section*{Problem 4.}
    Find two linearly independent power-series solutions of
    \[
      (1 - x^2)y''(x) - xy'(x) + p^2y(x) = 0
    \]
    where $p$ is a constant. Show that if $p$ is a non-negative
    integer, there is a solution that is a polynomial of degree $p$.

  \hrulefill

  \section*{Solution:}

  \clearpage

\end{document}
