\documentclass[rblock]{fkpset}

% I'm gonna include hyperlinks to websites in this document, ergo
% I load in the hyperref package (which adds support for doing this).
% If you're using this for psets and the like, you shouldn't include
% this without reason.
\usepackage{hyperref}
\hypersetup{
    unicode=false,                   % non-Latin characters in Acrobat’s bookmarks
    pdftitle={Example fkpset usage}, % title
    pdfauthor={Forest Kobayashi},    % author
    colorlinks=true,                 % false: boxed links; true: colored links
    urlcolor=blue                    % color of external links
}



% Make this appear with the rblock command. Be warned, you might have to
% manually adjust the positioning of the points table if you do so.
\name{Forest Kobayashi}
\class{Math 174}
\duedate{1/30/2019}
\assignment{Homework 1}

\problems{1.3, 1.5, 1.7(abc), 1.13, 1.15}

\lfoot{Due January 30th, 2019}
\chead{Math 174 --- HW \#1}
\rhead{Spring 2019}

\begin{document}
  % Syntax: \pointstable[alignment]{size}
  %
  % alignment -- controls positioning (e.g., centering, flushleft, etc)
  % size -- `mandatory' argument (also can just be left blank) --- controls the
  % size of the text in the table
  \pointstable{}

  % Example of how to generate points table manually
  % \vspace{-2.3cm}
  % \begin{table}[h]
  %   \centering
  %   \begin{tabular}{@{}lcccr@{}} % booktabs points table
  %       \toprule
  %       Problem & 1 & 2 & 3 & Total \\ \midrule
  %       Points  &   &   &   & \\ \bottomrule
  %   \end{tabular}
  % \end{table}
  % \vspace{1.5cm}

% --------------------------- Problem 1 ---------------------------- %

  \begin{problem}[1 (Sagan 1.13)]
    Here's an example of some things you can do with my pset class,
    and the associated packages it loads in.
  \end{problem}

  \begin{solution}
    \begin{enumerate}
      \item Make an enumerated list! The default labeling scheme is
        paren-wrapped letters. But because I load in
        \texttt{enumitem}, you can also make things like
        \begin{enumerate}[label=(\arabic*)]
          \item A list with paren-wrapped numbers
        \end{enumerate}
        \begin{enumerate}[label=\roman*)]
          \item A list with lower-case roman letters, with parens on
            the right-hand side
          \item See
            \href{https://www.overleaf.com/learn/latex/Lists}{here}
            for more.
        \end{enumerate}
      \item {\color{red} you can make text different {\color{blue}
            colors!}}
    \end{enumerate}
  \end{solution}
  \begin{solution}[You can change the text displayed at the start of
    solutions!]
    Woohoo!

    You can also change the QED symbol:
    \renewcommand{\qed}{\hfill $\triangle$}
  \end{solution}
  \begin{solution}
    \ldots As well as doing a ton of cool math stuff. Let's see some
    examples:
    \begin{align*}
      f(x)
      &= \text{you can align things!} \\
      &= \frac{1}{3x^2 + \sinh\pn{\frac{1}{2x}}}
        \shortintertext{and interrupt align environments briefly for an aside\ldots}
      &= g\pn{\exp\pn{kx^{3/2}}}
    \end{align*}
    before resuming again! You can do casework
    \begin{enumerate}[label=(\arabic*)]
      \item Case 1: [\ldots] \cmark
      \item Case 2: [\ldots] \cmark
    \end{enumerate}
    you can cancel things in equations:
    \[
      \frac{1}{\omega \bcancel{\kappa\eta^2}}
      \frac{\cancel{\kappa\eta^2}}{x^2}
    \]
    note that \verb|\cancel| and \verb|\bcancel| go in different
    directions. There's always things like \verb|\cancelto|
    \begin{align}
      \int_{t_1}^{t_2} \pn{\pd{\mc L}{\ddot{x}} \ddot{\eta}(t)} \dd t
      &= \cancelto{0}{\dot{\eta}(t) \pd{\mc L}{\ddot x}
        \bigg|_{t_1}^{t_2}} - \int_{t_1}^{t_2} \dot{\eta}(t)
        \pn{\od{}{t} \pn{\pd{\mc L}{\ddot{x}}}}\dd t \nonumber \\
      &= - \int_{t_1}^{t_2} \dot{\eta}(t) \bk{ \od{}{t}
        \pn{ \pd{\mc L}{\ddot{x}}}} \dd t \nonumber \\
      &= - \pn{\cancelto{0}{\eta(t)\pd{\mc L}{\ddot{x}}
        \bigg|_{t_1}^{t_2}} - \int_{t_1}^{t_2} \eta(t) \bk{\od[2]{}{t}
        \pn{\pd{\mc L}{\ddot x}}}} \dd t
        \nonumber \\
      &= \boxed{\int_{t_1}^{t_2} \eta(t) \bk{\od[2]{}{t}
        \pn{\pd{\mc L}{\ddot x}}} \dd t} \label{eq:tf el}
    \end{align}
    some other stuff I did in this equation:
    \begin{itemize}
      \item Use my auto-adjustable-sized-delimeters! There's a lot of
        these. The coolest one is \verb|\MID|, which auto-sizes to
        match the surrounding delimeters. E.g.,
        \[
          \mc F = \set{f(x) \in C[0,1] \MID f\pn{\frac{1}{2}} = e}
        \]
      \item Use my partial / total derivative shortcuts. The syntax is
        like
        \begin{center}
          \verb|\od[<order>]{<function>}{<variable of differentiation>}|
        \end{center}
        for ordinary derivatives, e.g.
        \[
          \od[6]{f}{x} \qquad \qquad \pd[2]{g}{y}
        \]
      \item Use my mathcal shortcut macro:
        \[
          \mc{ABCDEFGHIJLMNOPQRSTUVWXYZ}
        \]
        there are also some for other math fonts, such as blackboard bold:
        \[
          \mbb{ABCDEFGHIJLMNOPQRSTUVWXYZ}
        \]
        script font:
        \[
          \ms{ABCDEFGHIJLMNOPQRSTUVWXYZ} \qquad \ms{abcdefghijlmnopqrstuvwxyz}
        \]
        boldface:\footnote{Note, things like $\Aa$ also work for
          typesetting vectors quickly}
        \[
          \mb{ABCDEFGHIJLMNOPQRSTUVWXYZ} \qquad \mb{abcdefghijlmnopqrstuvwxyz}
        \]
        sans-serif font:
        \[
          \msf{ABCDEFGHIJLMNOPQRSTUVWXYZ} \qquad \msf{abcdefghijlmnopqrstuvwxyz}
        \]
        math roman:
        \[
          \mrm{ABCDEFGHIJLMNOPQRSTUVWXYZ} \qquad \mrm{abcdefghijlmnopqrstuvwxyz}
        \]
        as well as special macros for things that come up often, such
        as
        \begin{itemize}
          \item $\FF, \CC, \RR, \QQ, \ZZ, \NN$ for fields, complex
            numbers, reals, rationals, integers, and naturals
          \item $\PP$ for probability of an event, $\EE$ for expected
            value of a random variable
          \item $\TT$ for tori, $\DD$ for cubes, $\sS$ for the
            standard $n$-sphere (this macro is a bit of an anomaly,
            because \verb|\SS| was taken already)
        \end{itemize}
      And also you can automate referencing equations and things, such
      as Equation (\ref{eq:tf el})
    \end{itemize}
  \end{solution}
  \clearpage

  \begin{problem}[2]
    There are more things you can do!
  \end{problem}
  \begin{solution}
    \begin{leftbar}
      Hi look it's a leftbar environment!
      \begin{lemma}
        Oh wow a lemma! I love these!
      \end{lemma}
    \end{leftbar}
    Or an $A \iff B$ proof:
    \begin{iffproof}
      \item Suppose $A$. Then [\ldots]
      \item Suppose $B$. Then [\ldots]
    \end{iffproof}
    Similarly with set equality proofs:
    \begin{seteqproof}
      \item WTS $A \subseteq B$.
      \item WTS $A \subseteq B$.
    \end{seteqproof}
    as well as induction:
    \begin{induction}
      \item Notice $P(1)$. \cmark
      \item Let $k \in \NN$, and suppose $P(k)$. \cmark
      \item Then [\ldots], hence $P(k+1)$. \cmark
    \end{induction}
  \item Easy denoting of surjective / injective functions!
    \begin{itemize}
      \item $f : A \into B$
      \item $f : A \onto B$
      \item $f : A \bij B$
    \end{itemize}
  \item Also, image, preimage, and so on.
    \begin{align*}
      f\fim[A]
      &= g\fpre[A] \\
      \im f
      &= \pre g
    \end{align*}
  \item There's more stuff in the source. Some highlights are some
    integration things
    \[
      \lowint f(x) \dd x = \upint f(x) \dd x
    \]
    some big operators
    \[
      \osum_{i=1}^k V^{(i)} \qquad \qquad \oprod_{i=1}^{k} V^{(i)}
      \qquad \qquad \csum_{i=1}^k X_i
    \]
  \end{solution}
\end{document}